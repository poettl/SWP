\chapter{Betriebspraxisphase Überblick}

\begin{table}[!htbp]
	\caption{Betriebspraxisphase Überblick.}\label{tab3}
	\centering
	\begin{tabular}{|m{0.5\linewidth} | m{0.5\linewidth} |}\hline
		Studierendenname                  & Peter Öttl                                      \\\hline
		Email                             & oettl.peter@icloud.com                          \\\hline
		Partnerunternehmen Name           & aaa - all about apps GmbH                        \\\hline
		Partnerunternehmen Adresse        & Siebenbrunnengasse 17, 1050 Wien                \\\hline
		Partnerunternehmen Logo           & \includegraphics[width=0.6\linewidth]{PICs/aaa} \\\hline
		Partnerunternehmen Ansprechperson & DI Mario Ranftl, BSc\newline                    
		Email: \href{mailto:mario.ranftl@allaboutapps.at}{mario.ranftl@allaboutapps.at} }\newline
		Telefonnummer: +436642266295\\\hline
	\end{tabular}
\end{table}

\chapter{Einleitung}
\section{Beschreibung des Partnerunternehmens}

\grqq{}all about apps unterstützt Unternehmen während des gesamten Lebenszyklus ihrer digitalen Produkte. Wir begleiten Startups von der ersten Idee bis zur Veröffentlichung in den Stores. Aber auch traditionelle Unternehmen vertrauen auf die langjährige Erfahrung bei der Realisierung hochwertiger Mobile Solutions in den Bereichen Medizin, Industrie, Handel sowie Banken und Versicherungen\grqq{}\cite{aaa}.

\section{Definition Web Developer}

\grqq{}Ein Web-Entwickler ist zuständig für die Planung und Entwicklung von Webanwendungen in Unternehmen und ebenso für die Betreuung und Wartung von Websites. Ein Web-Entwickler koordiniert alle Aufgaben rund um die Planung zur Entwicklung von webbasierten Softwarelösungen und ist außerdem verantwortlich für das Design und die Grafik einer Website. Dazu gehört auch die Auseinandersetzung im Online Marketing in den Bereichen des Social Media und der Suchmaschinenoptimierung. Hier arbeitet ein Web-Entwickler eng mit dem Business Analysten, dem Content Manager und dem Social Media Marketing Manager zusammen\grqq{} \cite{webdev}.

\section{Arbeitsplatzbeschreibung}

Es wird mit einem MacBook Pro 16 gearbeitet, das an einem 24-Zoll Monitor im Großraumbüro der Firma angeschlossen wird. Diverse Büromaterialien sowie Kopfhörer u.ä. befinden sich auf dem Arbeitsplatz. Für die Verpflegung ist sowohl das Webrestaurant als auch das Unternehmen “Rita bringt’s” zuständig, das täglich frisch zubereitetes Bio-Essen liefert. Weiter gibt es für zwischendurch eine Kaffeeküche inkl. Snackbereich als auch Obstkörbe am Mittagstisch. Die Arbeitszeit wird selbstständig eingeteilt; Gleitarbeitszeit und Home-Office werden angeboten. Diese Freiheiten sorgen meiner Meinung nach für eine gute Work-Life Balance. Coronabedingt durften alle MitarbeiterInnen gänzlich im Homeoffice arbeiten.

\section{Inhaltliche Themen der Betriebspraxisphase}
%TODO
Wie auch bei den letzten Betriebspraxisphasen ging es auch in dieser um Web Entwicklung. Ein wichtiger Unterschied zu den letzten Arbeitsphasen wurde in dieser viel mehr auf die Backend Entwicklung gelegt. Hierbei habe ich mich mit Go\cite{go}\cite{gostarter}, PostgresSQL\cite{psql}, Docker\cite{docker} und Swagger\cite{goswagger}\cite{swagger} gearbeitet. Aber natürlich gab es auch kleinere Aufgaben im Frontendbereich wo, ich mit Technologien wie React\cite{react}, Firebase\cite{firebase} und Angular\cite{angular} gearbeitet habe. Außerdem konnte ich einen sehr wertvollen Einblick in die Cloud Infrastruktur von allaboutapps erhalten. 

\section{Lerntagebuch}

Im Lerntagebuch dokumentiere ich meine gesammelten Arbeitserfahrungen und die daraus neu gewonnenen Erkenntnisse, die ich in der Praxisphase sammeln durfte. Die Kapiteln sind in Wochen und Tagen unterteilt. Zum Schluss reflektiere ich meine Erlebnisse bezüglich der Projektarbeit, der Meetings, des Zeitmanagements, der Teamentwicklung und der neuen Erfahrungen.

\section{Corona Arbeitsmodus}

Seit Beginn der Betriebspraxisphase im Februar 2021 haben alle MitarbeiterInnen im Homeoffice gearbeitet. Ab März 2021 durften wir nach negativ getesteten Homekits, die von allaboutapps zur Verfügung gestellt wurden, in das Büro kommen. Weiters wurde eine Shared und Clean Desk Policy eingeführt. In dieser bleibt der Monitor am Tisch stehen, während persönliche Gegenstände am Ende des Tages wieder entfernt werden müssen. Es muss im gesamten Büro eine FFP2 Maske getragen werden. Ab April 2021 werden Spinde für jede/n MitarbeiterInnen zur Verfügung gestellt.

\section{Arbeitsklima und gemeinsame Aktivitäten}

Ich habe in all meinen bisherigen beruflichen Erfahrungen noch nie so ein tolles Arbeitsklima und solch einen Zusammenhalt wie bei allaboutapps erlebt. Die MitarbeiterInnen interagieren miteinander wie eine Familie, in der man sich aufeinander verlassen kann und in welcher jeder jedem hilft. Meiner Meinung nach haben Events wie z.B. Afterwork, Sommerfest, Winterausflug, Weihnachtsfeier und weitere TeamEvents den Zusammenhalt gefördert, was wiederum die Arbeitsmotivation steigert. Wenn die MitarbeiterInnen sich nämlich gut kennen und gerne miteinander Zeit verbringen, fördert dies die Stimmung im Büro ungemein. Besonders haben mir die klein gehaltenen Release Feiern (nach abgeschlossenen Projekten) gefallen; aufgrund der Covid Bestimmungen sind diese nämlich nicht ausgefallen, sondern wurden von zu Hause aus online gehalten, während jeder Essen und Getränke sich hat nach Hause liefern lassen.

\chapter{Projektbeschreibungen}

\section{Projekt A}

An Projekt A durfte ich schon während meiner letzten Betriebspraxisphasen arbeiten. Hierbei geht es um eine Angular\cite{angular} App, die von einem Smartphone aufgezeichnet, Musik visualisiert sowie genauere Informationen über die gespielten Lieder bereitstellt. Der User, in diesem Fall z.B. ein DJ, kann in einer bereitgestellten Statistik erkennen, welche/r UserIn welche Lieder abspielt. Es ist ebenso ersichtlich, welche Lieder bei früheren Events abgespielt wurden. Das Projekt wird unter anderem mittels Google Geochart und Firebase\cite{firebase} umgesetzt.

\section{Projekt B}

Projekt B ist sehr stark mit Projekt A verbunden. Hierbei handelt sich um ein firmeninternes Tool mit dem die Aufzeichnung der Stream organisiert wird. Bei diesen Streams handelt es sich z.B. um Twitch, YouTube und Facebook-Streams. Bei diesen Streams wird die Musik aufgezeichnet und kann mittels eines Widgets, das beim Stream eingebunden werden kann, angezeigt. Für die Musikerkennung wird hier auf Firebase\cite{firebase} zugegriffen, während die Streams von einem externen Backend kommen.

\section{Projekt C}

Bei dem Projekt C geht es um einen Konfigurator für Klimaanlagen, für eine große deutschsprachige Elektronikmarkt-Kette. Der/Die VerbraucherIn kann durch die Angaben von Immobilientyp, Größe der Zimmer, Mauertyp und gewünschte Position im Raum sich beraten lassen, welche Klimageräte, Außengeräte, Länge der Leitungen usw. am geeignetsten wären. Zudem wird ein CMS benötigt bei dem alle gesammelten Bestellungen anzeigt und verwaltet werden. Bei diesem Projekt habe ich zum erste Mal im Backend Team arbeiten dürfen. Der Backend Stack besteht aus Go\cite{gostarter}\cite{go} und PostgresSQL\cite{psql}.

\section{Projekt D}

Projekt D ist eine React App mit der Hausverwaltungen oder Mieter Schlüssel für Immobilien nachbestellen können. Die Bestellungen müssen daraufhin von der jeweiligen Hausverwaltung genehmigt werden. Die unterschiedlichen Zahlungsoptionen, die zur Zeit nicht miteinander abgestimmt werden können, sind bei diesem Projekt ein große Herausforderung. Der aktuelle Lösungsansatz ist, dass dies mit hoher Wahrscheinlichkeit durch Stripe gelöst werden kann. Dieses Projekt ist mein erstes großes React\cite{react} Projekt, da ich bisher fast ausschließlich mit Angular gearbeitet habe. An diesem Projekt habe ich schon in der letzten Betriebspraxisphase gearbeitet.


\chapter{KW 5}

\section{Montag, 01. Februar 2021}
Heute startet meine dritte Betriebspraxisphase. In dieser Praxisphase wird sich der Großteil um die Backend Entwicklung mit Go\cite{gostarter}\cite{go} handeln. Darum starte ich heute mit dem Lernen der Sprache Go. Aus diesem Grund habe ich mich mit einem Test Projekt\cite{beerPunk} auseinandergesetzt, das eine gute Grundlage für den allaboutapps Backend Stack ist. Am Nachmittag gab es wieder ein Sprintmeeting für das Projekt A/B. Hierbei wurde die kommende Woche für Frontend und Backend geplant.

\section{Dienstag, 02. Februar 2021}
Manche Tracks werden manchmal mit der API nicht erkannt, die aber bei Live Streams dringend benötigt werden. Beim Projekt B habe ich deshalb einen Dialog implementiert, mit dem Tracks erstellt werden können, sowie einem Live Stream zugewiesen. 

\section{Mittwoch, 03. Februar 2021}
Heute hab ich bei Projekt B das Design der Dialoge ausgelagert damit es bei allen Komponenten angewendet werden kann. Zusätzlich habe ich mich weiter mit Go beschäftigt.

\section{Donnerstag, 04. Februar 2021}
Bei Projekt A habe ich den Eventdialog angepasst um auch wiederkehrende Events zu unterstützen. Außerdem musste ein Bug behoben werden, damit bereits generierte wiederkehrende Events gelöscht werden können.

\section{Freitag, 05. Februar 2021}
Heute hatte ich mein erstes Web Team Meeting und das Standup. Die restliche Zeit habe ich mich mit Go beschäftigt.


\chapter{KW 6}

\section{Montag, 08. Februar 2021}
Heute wurde ein Intro Meeting mit dem Projektmanager von Projekt C abgehalten. Er hat mir das Projekt erklärt und mich über meine Aufgabenbereiche informiert. Am Nachmittag hatte ich ein weiteres Sprintmeeting für Projekt A/B. In der restlichen Zeit habe mich erneut mit Go beschäftigt.

\section{Dienstag, 09. Februar 2021}
Beim Projekt C hatten wir heute das erste Sprintmeeting. Meine Aufgabe war es, einen Endpunkt für die CMS Übersicht und Bestelldetails zu erstellen. Zusätzlich habe ich den Template Dialog im Projekt B angepasst und mich mit User Claims beschäftigt, damit unterschiedliche Benutzerrechte implementiert werden können.

\section{Mittwoch, 10. Februar 2021}
Heute habe ich beim Rechte-Management weitergearbeitet. Beim Login wird geprüft, welche Rechte die/der jeweilige BenutzerIn hat. Anhand der Rechte wird die Seite aufgebaut. Ich habe zudem Guards hinzugefügt, da manche Routen nur mit bestimmten Benutzergruppen möglich sind. Es gab des Weiteren eine Retrospektive von einem alten Projekt. In diesem sagte jede/r Beteiligte, was gut und was nicht so gut gelaufen ist, sowie was beim nächsten Mal besser gemacht werden könnte.

\section{Donnerstag, 11. Februar 2021}
Gemeinsam mit Michael haben wir heute eine Entwicklungsumgebung im Kubernetes Cluster umgesetzt, ein ERM entwickelt und mit der Entwicklung gestartet mit der Produktroute.

\section{Freitag, 12. Februar 2021}
Wie jede Woche gab es heute ein Web Team Meeting und ein Standup. Ansonsten habe ich mithilfe des ERMs die SQL Migrations geschrieben.


\chapter{KW 7}

\section{Montag, 15. Februar 2021}
Heute hatte ich das Sprintmeeting bei Projekt A/B und ein Meeting bei Projekt C. Ich habe dann beim Template Feature für Projekt B weitergearbeitet.

\section{Dienstag, 16. Februar 2021}
Bei Projekt A wird ein Email Login benötigt. Aktuell wird Facebook und die dazugehörigen Artists-Seiten für den Login Prozess benötigt. Da viele UserInnen via Feedback einen alternativen Prozess vorgeschlagen.

\section{Mittwoch, 17. Februar 2021}
Heute habe ich beim Email Login weitergearbeitet. Dazu habe ich die Firebase Auth Funktionen für "Login", "Registrieren" und "Passwort vergessen" im Projekt angewendet.

\section{Donnerstag, 18. Februar 2021}
Bei Projekt A/B hatten wir ein Intro Meeting mit Matthias. Er wird langfristig das Projekt weiterentwickeln. Außerdem habe ich mich mit den swagger definition für den postOrderDuplicate handler beschäftigt.

\section{Freitag, 19. Februar 2021}
Wie jede Woche fand heute ein Web Team Meeting und ein Standup statt. Ansonsten habe ich mich weiter mit dem postOrderDuplicate handler beschäftigt.


\chapter{KW 8}

\section{Montag, 22. Februar 2021}
Für das Projekt A und Projekt C hatten wir wie jede Woche unser Sprintmeeting. Mit Michael hatte ich ein weiteres Meeting bezüglich der Typisierung bei Go. Auch habe ich mich mit dem getOrderDetails handler beschäftigt.

\section{Dienstag, 23. Februar 2021}
Bei den Orders habe ich die Migrations angepasst und bei dem getOrders Handler weitergearbeitet. Zusätzlich habe ich Matthias beim Projekt A unterstützt. Am Abend hatten wir unser Release Bier Event für das Projekt C.

\section{Mittwoch, 24. Februar 2021}
Heute hatte ich ein Meeting für das Projekt A/B bezüglich dem Email Login und das weitere Vorgehen. Am Nachmittag hatten wir einen Go Workshop mit Mario. Es wurden viele Themen des go-starter\cite{gostarter} behandelt, wie z.B. SQL Migrations, Swagger, Routing Handler, Testing, SQL First und Swagger First. Beim Projekt C habe ich weitere Tests für den Handler von OrderDuplicate implementiert und die Produkt Migrations angepasst.

\section{Donnerstag, 25. Februar 2021}
Bei Projekt A habe ich mich heute mit Email Login, Email Verifizierung und Account Verbindung auseinandergesetzt. Zusätzlich habe ich Benutzer Claims für Projekt B implementiert.

\section{Freitag, 26. Februar 2021}
Wie jede Woche hatten wir heute das Web Team Meeting und das Standup. Beim Projekt C habe ich eine CMS Route und einen Importer für Produkte hinzugefügt. Zudem habe ich mit dem CSV Importer gestartet. Beim Projekt A hatten wir ein weiteres Meeting über das Account Linking.


\chapter{KW 9}

\section{Montag, 01. März 2021}
Heute hatten wir ein Sprintmeeting für das Projekt A/B und C. Bei Projekt C haben wir uns über die Produkt Sets und den Importer gesprochen. Beim Importer werden wir von CSV auf Excel umsteigen. Hierfür werde ich das Paket excelize\cite{excelize} verwenden. Außerdem habe ich ein neues Macbook 16 bekommen.

\section{Dienstag, 02. März 2021}
Bei Projekt A hatten wir ein Webmeeting bezüglich Account Merge und Login mit Twitter, Facebook und Instagram. Außerdem habe ich beim Projekt C beim Produktimporter weitergearbeitet, Transactions hinzugefügt und mich mit Upsert beschäftigt.

\section{Mittwoch, 03. März 2021}
Bei Projekt C habe ich beim Importer weitergearbeitet und hierfür Tests geschrieben. Am Nachmittag hatten wir unseren zweiten Teil des Go Workshops bei welchen Beer Punk API implementiert haben.

\section{Donnerstag, 04. März 2021}
Heute habe ich mich mit der Facebook Graph API und Claim Artist beschäftigt. Zusätzlich hatten wir einen Flutterworkshop wo wir uns mit dem Anbinden der Beer API, Listen und Bilder anzeigen implementiert haben.

\section{Freitag, 05. März 2021}
Neben dem Web Team Meeting und dem Standup habe ich den Login und Signup Screen überarbeitet und das Design verbessert.


\chapter{KW 10}

\section{Montag, 08. März 2021}
Heute hatten wir wieder ein Sprintmeeting für das Projekt A/B und C. Bei Projekt C habe ich beim Excel Importer für die Produkte und Sets weitergearbeitet.

\section{Dienstag, 09. März 2021}
Heute habe ich den Importer für die Produkte, Sets und Bilder fertiggestellt. Außerdem habe ich den Produkt Importer dahingehend erweitert, damit sicherstellt ist, dass das dazugehörige Bild tatsächlich existiert.

\section{Mittwoch, 10. März 2021}
Bei Projekt A habe ich heute eine Überprüfung eingebaut, mit der überprüft wird, ob der aktuelle Artist bereits verifiziert ist. Zusätzlich habe ich mich mit den Firebase Regeln und Tests für den Importer beschäftigt.

\section{Donnerstag, 11. März 2021}
Heute habe ich den Claim Artist Dialog überarbeitet, damit die Logik auch ohne einem Dialog funktioniert. Am Nachmittag fand der zweite Teil unseres Flutterworkshops statt. In diesem haben wir uns mit dem Detail Screen, Google Maps und API Aufrufe beschäftigt.

\section{Freitag, 12. März 2021}
Abgesehen von dem Web Team Meeting und dem Standup, habe ich mich mit der Produktmatrix Logik beschäftigt.


\chapter{KW 11}

\section{Montag, 15. März 2021}
Heute hatten wir wieder das Sprintmeeting für das Projekt A und C. Bei Projekt A hatte ich heute die finale Übergabe an Matthias. Bei Projekt C habe ich mich mit der Produktmatrix und der Validierung, der Sortierung und dem Filter von Bestellungen beschäftigt.

\section{Dienstag, 16. März 2021}
Heute habe ich mich mit den Parametern für Filter, Sortierspalten und dem Sortierstatus auseinandergesetzt. Ich habe zudem \dq LIKE\dq{} und \dq OR\dq{} für die \dq Where\dq{} - Abfrage erstellt. Schließlich habe ich an der Produktmatrixlogik weitergearbeitet.

\section{Mittwoch, 17. März 2021}
Bei Projekt C arbeitete ich an der Bildserver-Route, um die Möglichkeit zu haben ohne Handler auf die Bilder zugreifen zu können. Später fand ein Meeting für das Projekt A bezüglich dem Claim Artist pageID Problem statt.

\section{Donnerstag, 18. März 2021}
Bei Produkt-Importer habe ich kleine Anpassung implementiert, aus dem Grund, dass die Excel-Datei sich beim Kunden geändert hatte. Bei Bilder-Importer habe ich das Problem mit copyN und copy gelöst. Es wurde zudem ein Meeting für die Vorbereitung der Firmenmesse im FH Technikum abgehalten.

\section{Freitag, 19. März 2021}
Bei dem Set-Importer wurden Änderungen fällig, da sich auch für hier die Excel-Datei geändert hatte. Zusätzlich hatten wir eine Umstellung mit Jira. Allaboutapps hat sich dafür entschieden, von einer selbst gehosteten Lösung zur einer Cloud-Lösung zu wechseln. Das Webteam Meeting und das Standup fanden statt. Die übrige Zeit habe ich mich mit dem Sets Handler beschäftigt.


\chapter{KW 12}

\section{Montag, 22. März 2021}
Heute hatten wir das wöchentliche Sprintmeeting für das Projekt C. Beim Bildimporter habe ich die URL angepasst. Der relative Pfad wird nun in der Datenbank gespeichert und der Rest beim Handler hinzugefügt. Außerdem habe ich den Produktimporter um zwei weitere Felder ergänzt.

\section{Dienstag, 23. März 2021}
Ich habe heute bei der Produktmatrix weitergearbeitet und mich den Produkten beschäftigt, welche für die inneren Räume und welche als Außengeräte passen könnten.

\section{Mittwoch, 24. März 2021}
Da für eine Liste alle Produkte benötigt werden, habe ich den Limitparameter optional gemacht. Weiter habe ich einen Fehler beim Sortieren ausgebessert.

\section{Donnerstag, 25. März 2021}
Der Kunde wollte einige Änderungen. Daraufhin habe ich den Importer anpassen müssen. Zudem habe ich mich weiter mit dem Sets Endpunkt auseinandergesetzt. Mit diesem werden nun Set-Vorschläge für ausgewählte Produkte gemacht. Am Nachmittag war ich bei der FH Firmenmesse anwesend. Dort durfte ich Fragen bezüglich allaboutapps und dem dualen Studium an Interessierte beantworten.

\section{Freitag, 26. März 2021}
Neben dem Webteam Meeting habe ich mich mit dem Sets Endpunkt beschäftigt. Auch habe ich den aktuellen Stand in der Testumgebung bereitgestellt.


\chapter{KW 13}

\section{Montag, 29. März 2021}
Wie jede Woche hatten wir heute das Sprintmeeting für das Projekt C. Beim Set Endpunkt habe ich die Logik erweitert und dazu Tests geschrieben.

\section{Dienstag, 30. März 2021}
Beim Produktdetails Endpunkt habe ich heute die Eigenschaften angepasst. Außerdem habe ich den Handler für die Produktmatrix implementiert.

\section{Mittwoch, 31. März 2021}
Beim Gesamtwert für den Bestellung Endpunkt habe ich einen Fehler gelöst, denn es wurde immer nur maximal das Limit als Gesamtwert zurückgeben und nicht der gesamte Gesamtwert. Auch habe ich den Handler und die Tests für die Produktmatrix fertiggestellt. Am Nachmittag hatte ich ein Meeting mit Mario bezüglich Kubernetes. Ich bekam eine kurze Einführung und daraufhin Zugriff in die allaboutapps Infrastruktur. Der Grund für den Zugriff war, dass ich auf die einzelnen Logdateien Einsicht bekomme und bei Fehlern leichter eine Lösung finden kann.

\section{Donnerstag, 01. April 2021}
Heute habe ich das Versenden mit dem lokalen Mailserver Mailhog getestet. Beim Set habe ich die Verbindung zwischen Produkt und Set angepasst.

\section{Freitag, 02. April 2021}
Neben dem Set Handler, dem Webteam Meeting, dem Standup habe ich an den Email Vorlagen gearbeitet.


\chapter{KW 13}

\section{Montag, 05. April 2021}
Feiertag

\section{Dienstag, 06. April 2021}
Das Projekt C Sprintmeeting fand heute statt, da Montag ein Feiertag war. Ich habe mich mit dem Email Versand und der internen Notifikation auseinandergesetzt. Zudem habe ich einen Endpunkt für die Berechnung des Preises entwickelt.

\section{Mittwoch, 07. April 2021}
Es wurde für den abschließenden Bestellungs-Endpunkt die Logik für die Berechnung des Preises hinzugefügt. Bei der Kubernetes Entwicklungsumgebung habe ich die STMP Einstellungen erstellt. 

\section{Donnerstag, 08. April 2021}
Heute habe ich an der Validierung der Bestellung gearbeitet. Wir hatten eine Besprechung über neue Funktionen, wie z.B. bezüglich der PDF Anhänge für eine E-Mail, Markierung ausverkaufter Produkte und Deaktivierung von Konfigurationen. Im Datenmodell habe ich für den Haustyp und Raumtyp jeweilige fixe Typen hinzugefügt.

\section{Freitag, 09. April 2021}
Neben dem Webteam Meeting und Standup habe ich mich heute mit dem Google SMTP Login Problem auseinandergesetzt. Google unterstützt den normalen Login mit Email und Passwort nämlich nicht mehr. Für einen Login ohne OAuth muss nun ein App-Passwort generiert und die zwei Faktor Zwei-Faktor-Authentifizierung aktiviert werden.


\chapter{KW 14}

\section{Montag, 12. April 2021}
Beim Sprintmeeting für das Projekt C haben wir über die kommende Woche gesprochen. Ich habe mich zudem mit dem Nachladen von Datenbank-Referenzobjekten mithilfe von SQL Boiler beschäftigt. Für die Bestelldetails wurde ein extra Endpunkt benötigt, damit die Daten auch mit der Konfigurationsnummer geladen werden können.

\section{Dienstag, 13. April 2021}
Heute habe ich an den Order Snapshots gearbeitet. Hierzu erstellte ich aus dem Objekt ein JSON und speicherte dies extra ab. Zusätzlich musste beim Laden der Daten berücksichtig werden, ob ein Snapshot erstellt und gespeichert wird. Bei Projekt D hatte ich heute das Intro Meeting. Ich habe erfahren welche Funktionen ich implementieren soll.

\section{Mittwoch, 14. April 2021}
Neben dem Sprintmeeting für das Projekt D habe ich mich zusätzlich mit einem Problem bezüglich der Produktnamen von den sonstigen Leistungen beschäftigt. Ich habe beim Rechte-Management gekennzeichnet, welche Benutzer die Produktmatrix überschreiben dürfen. Ich fügte noch weitere kleinere Änderungen in der Swagger Datei zu.

\section{Donnerstag, 15. April 2021}
Bei Projekt C habe ich Fehler im Berechnen des Preises gelöst. Ansonsten habe ich bei Projekt D kleine Fehler bei den Schriften der Ladeanzeige und der Einstellungsseite beseitigt.

\section{Freitag, 16. April 2021}
Neben dem Web Team Meeting und beim Dev Standup habe ich bei Projekt D den Benutzerhinzufügten Dialog und die Benutzeranzeige implementiert.


\chapter{KW 15}

\section{Montag, 19. April 2021}
Heute habe ich bei Projekt D mit dem Google Maps SDK arbeiten dürfen. Hier habe ich eine Autovervollständigung implementiert, in dieser der eigene Standort gesucht und angezeigt werden kann. Nach der Auswahl wird im Umkreis von 1000 Metern nach allen Schlüsseldiensten gesucht sowie in der Karte visualisiert. Eine Emailvorschau wurde implementiert, um einen Emailtext zu kopieren oder direkt das Emailprogramm mit dem Text zu öffnen.

\section{Dienstag, 23. April 2021}
Bei den Dashboardinfos habe ich den API-Aufruf angepasst. Nun wird die Rechnungsadresse, die Lieferadresse und die UID angezeigt und alles kann bearbeitet werden. Dann habe ich mich mit Anpassungen für die mobile Ansicht auseinandergesetzt. Bei der Google Maps Karte habe ich das Design angepasst und eigene Bilder für den Marker hinzugefügt, damit alles besser zum restlichen Design passt.

\section{Mittwoch, 24. April 2021}
Bei der Objektliste habe ich den Benutzer Endpunkt angebunden. Für Projekt C wurde ein Meeting abgehalten, wo die restlichen Fehler, die noch behoben werden müssen, besprochen wurden. Das Email Template und der Importer für die Produkte sowie den Bildern wurde angepasst.

\section{Donnerstag, 25. April 2021}
Bei Projekt D habe ich eine Logik für Rabattcodes implementiert. Die Anbindungen für die Objekte und Benutzer wurden zudem fertiggestellt.

\section{Freitag, 26. April 2021}
Heute war mein letzter Arbeitstag und ich habe dokumentiert, welche Sachen nicht fertig wurden und wo sich diese befinden. Außerdem hatte ich mein letztes Web Team Meeting und das Standup Meeting.


\chapter {Reflexion}

\section{Learnings}

Wie auch in den letzten Betriebspraxisphase, durfte ich in dieser sehr vieles lernen. Es hat mir besonders gefallen, dass ich ein Teil von besonders interessanten Projekten sein durfte. Die größte Fortschritte in dieser Praxisphase hatte ich in der Backend Entwicklung. Das Go-Starter\cite{gostarter} Backend Template ist, meiner Meinung nach eine tolle Möglichkeit, um API Anwendungen zu implementieren. Besonders fiel mir auf, dass mit der Automatisierung von Code Generierung sehr effizient entwickelt werden kann. Neben Go habe ich mich mit Swagger, Postgres und API Architektur beschäftigen dürfen. \newline
Ich konnte erneut an Frontend Projekten mitarbeiten. Mein Highlight hier war definitiv die Google Maps Integration in ein React\cite{react} Projekt mithilfe des Google Maps Javascript SDK. Besonders interessiert hat mich die Einführung in die allaboutapps Infrastruktur. Es wird mit Google Cloud und Kubernetes gearbeitet. Ich freue mich bereits jetzt auf meine nächste Betriebspraxisphase im Juli 2021.

\section{Meetings}

Bei allaboutapps gibt es wöchentliche Teammeetings, in denen wichtige Neuerungen, Erkenntnisse oder auch firmeninterne Informationen an das Team weitergeleitet werden. Jede/r kommt nacheinander dran und teilt seine/ihre Erfahrungen, die er/sie während seiner/ihrer Arbeitswoche gemacht hat. Wenn sowohl die Erfahrungen als auch die angetroffenen Probleme mit der Gruppe geteilt werden, hilft dies allen, die eventuell in späterer Folge auf dieselben oder ähnlichen Problemen stoßen könnten, da bereits Lösungsansätze zur Behebung präsentiert wurden.\newline
Beim Standup Meeting kommen meist kurz nach Mittag alle Entwickler zusammen und berichten über ihre Aufgaben seit dem zuvor abgehaltenen Standup. Ich war immer sehr froh über diese kurzen und informativen Meetings, die mir den Austausch mit meinen Kolleg*innen leicht ermöglichten. Besonders gefällt mir die Anwendung Slack, die von den Mitarbeiter*innen verwendet wird, um alle auf den neuesten Stand über bestimmte Neuigkeiten zu halten; hiermit entsteht keine Zeitinvestition, in der alle Mitarbeiter*innen koordiniert werden müssen. Ein weiterer Punkt, den ich erwähnen möchte, ist, dass die Meetings seit Anfang der Covid Pandemie über Google Meet abgehalten werden. Das funktioniert sehr gut und ermöglicht es einem die Teammitglieder immer wieder zu sehen, obwohl man sich nicht miteinander trifft.

\section{Projektarbeit}

Im Unternehmen werden verschiedene Agile Methoden bei der Projektarbeit angewandt. Hier kommt unter anderem das Projektmanagement Tool Jira von Atlassian zum Einsatz, in dem die Sprints, Bugs, Tickets, Stories etc. verwaltet werden. Für die Dokumentation der Projekte wurde mit Confluence gearbeitet. \newline
Ein klassisches Projekt fängt für den Entwickler mit einer groben Aufwandseinschätzung an, damit der Projektmanager erste Abschätzungen bezüglich der Zeitinvestition als auch den auf den Kunden zukommenden Kosten mit dem Kunden selbst besprechen kann. Wenn das Projekt von Seite des Kunden wie besprochen genehmigt wird, beginnt das Kick Off Meeting. Im Kick Off Meeting werden alle Features und Besonderheiten im Entwicklungsteam detailliert besprochen. Hier haben alle Mitwirkende die Möglichkeit etwaige Fragen zu stellen. Danach starten die Sprintmeetings. In diesen werden die Features in kleine Pakete unterteilt, damit alle Beteiligten beim Entwickeln den Überblick behalten können. Auch fällt die Schätzung des zeitlichen Aufwandes damit leichter. Zu Anfang ist mir das Schätzen meiner Arbeitszeit etwas schwergefallen. Meine Einschätzungen wurden aber von Sprint zu Sprint wesentlich besser. Die Tickets haben meiner Meinung nach noch einen weiteren Vorteil. Ich wusste nämlich immer darüber Bescheid auf welchen Stand der Rest des Entwicklungsteam sich befindet und ob sich bis zum Sprintende alles termingerecht ausgehen wird oder nicht. Natürlich sind Planung und Umsetzung zwei verschiedene Dinge. Aufgrund von diversen Schwierigkeiten kommt es somit vor, dass sich nicht immer alle Sprints ausgehen. Im Team wurden etwaige Probleme, die ich hatte, immer direkt besprochen. Ich konnte mich durchgehend auf meine Teamkolleg*innen verlassen, die mir jederzeit kompetent zur Seite standen. Wenn mir manche der Tickets schwergefallen sind, war es mir jederzeit möglich, ein kurzes Meeting mit meinem Projektmanager auszumachen. In diesen wurden dann die jeweiligen Sprints so umgeplant, dass ich daraufhin gleich mit einem neuen Task anfangen konnte.\newline
Bei allaboutapps werden Teams je nach Projekt zusammengesetzt. Grundsätzlich gibt es einen Projektmanager und je nach Bedarf Backend, Web-Frontend, iOS und Android Developer. Die Leitung übernimmt der Projektmanager und teilt die Arbeit den einzelnen Entwicklern zu. Hierbei wird vor allem mit Sprintmeetings gearbeitet. Die Zusammenarbeit funktioniert sehr gut. Die meiste Interaktion passiert zwischen den Backend und Frontend Entwicklern, da diese beim Programmieren voneinander abhängig sind.

\section{Kommunikation bei allaboutapps}


\begin{figure}[!htbp]
\centering
\includegraphics[width=0.7\linewidth]{PICs/schulz}
\caption{Das Kommunikationsquadrat von Schulz von Thun}\label{Abb1}
\end{figure}

Das Kommunikationsquadrat („Vier-Ohren-Modell” oder „Nachrichtenquadrat”) ist ein bekanntes Modell von Friedemann Schulz von Thun, das besagt, dass Menschen beim Kommunizieren mit anderen sich immer über vier Botschaften äußern. Der Sender ist genauso für die Qualität der Kommunikation verantwortlich wie der Empfänger. \cite{schulzThun} \newline
Auf der Sachebene ist die Sachinformation besonders wichtig, denn hier geht es um Daten, Fakten sowie Sachverhalte. Für die Selbstkundgabe gilt: Jede Äußerung enthält gewollt oder unfreiwillig einen Teil der Persönlichkeit: Gefühle, Werte, Eigenarten und Bedürfnisse. Dies kann explizit („Ich-Botschaft”) oder implizit geschehen. Auf der Beziehungsseite gebe ich Informationen darüber, was ich vom anderen denke bzw. wie ich zu ihm stehe. Die Vermittlung der Beziehungshinweise wird durch Formulierung, Tonfall, Mimik und Gestik erzielt. Auf der Appellseite wird der Empfänger beeinflusst. Er äußert Wünsche, Appelle, Ratschläge oder Handlungsanweisungen. \cite{schulzThun} \newline
Im Bezug auf allaboutapps kann festgestellt werden, dass die Kommunikation, wenn über Projekte gesprochen wird, das Meiste in der Appellebene und in der Sachebene stattfindet. Die Beziehungsebene und die Ebene der Selbstkundgabe spielt eher eine Rolle, sobald gegenseitige Hilfe benötigt wird. Natürlich kommen während allen Gesprächen alle vier Ebenen vor, aber die jeweils oben angeführten Beispiele kommen hauptsächlich vor.





















